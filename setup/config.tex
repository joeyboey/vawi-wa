% Raender: 3cm Seite, 2 cm oben & unten
\usepackage{typearea}
\areaset{150mm}{257mm}
\usepackage[small,bf,up]{caption}  %nicer captions
\renewcommand{\captionfont}{\small}
\usepackage[hyphens]{url}
%\usepackage{picins}
\usepackage{ae}                 % Für PDF-Erstellung
\usepackage{latexsym}			% schönere Symbole
%\usepackage{booktabs}
\usepackage{xcolor}
\usepackage{float}
\usepackage{etex}

% Zeichensätze
\usepackage[utf8]{inputenc}
\usepackage[T1]{fontenc} % T1-kodierte Schriften, korrekte Trennmuster fuer Worte mit Umlauten
%\usepackage[iso]{umlaute}

% Meta-Informationen
\renewcommand\author{Autor}
\renewcommand\title{Titel}
\renewcommand\subtitle{Untertitel}
%\renewcommand\subject{Projekt-/Masterarbeit/Anrechnung beruflicher Kompetenzen\\
%im Studiengang Wirtschaftsinformatik der Fakultät Wirtschaftsinformatik und Angewandte Informatik der Otto-Friedrich-Universität Bamberg}
\renewcommand\date{xx.xx.xxxx}
   
% Hyperref
\usepackage{hyperref}
\definecolor{darkblue}{rgb}{0,.05,.54}
\definecolor{darkgreen}{rgb}{0,.54,.05}
\definecolor{vawiblue}{rgb}{.25,.48,.67}
\definecolor{vawigrayblue}{rgb}{.41,.47,.53}
\makeatletter
\hypersetup{pdftitle={\@title}, pdfauthor={\@author}, pdfsubject={\@subject}, colorlinks=true, breaklinks=true, linkcolor=black, menucolor=darkblue, urlcolor=darkblue, citecolor=darkblue, filecolor=darkblue}
\makeatother
\urlstyle{same}
\usepackage[all]{hypcap}


% Mathe und Formeln
\usepackage{calc}
\usepackage[centertags]{amsmath}
\usepackage{amssymb,amsthm,amsfonts}
\usepackage{cancel}  %%druchstreichen von Formeln

% Programmieren mit Latex
\usepackage{ifthen}

% Zeilenabstand
%%   1,1-facher Zeilenabstand   %%
\usepackage{setspace}
\setstretch{1.5}
%\setlength{\parindent}{0em}

\usepackage{dirtree}   %setzen von baumstrukturen
\usepackage{bbding}    % Hände

\usepackage{paralist}      % kleinere Anstände bei enums und itemizes möglich

%%   Fuer anspruchsvolle Tabellen   %%
\usepackage{longtable, colortbl}
\usepackage{array}
\newcolumntype{P}[1]{>{\raggedright\let\newline\\\arraybackslash\hspace{0pt}}p{#1}}
\usepackage{multicol, multirow}

%%  Für Grafiken %%
\usepackage[pdftex]{graphicx}
\usepackage{tikz}
\usepackage{pgfplots}
\usetikzlibrary{arrows,shapes,fit,positioning,decorations.pathmorphing,backgrounds,shadows}
%\tikzstyle{line} = [thick]

%% Listings %%
%% Code Highlighting
\definecolor{mygray}{gray}{.75}
\usepackage{listings} 
\lstset{numbers=left, numberstyle=\tiny, numbersep=6pt} 
\lstset{language=Python}
\lstset{classoffset=1, morekeywords={mycontext}, keywordstyle=\color{darkgreen}, classoffset=0, keywordstyle=\color{darkblue}}
\lstset{basicstyle=\small, showstringspaces=false, commentstyle=\color{mygray}, breaklines=true, captionpos=b}

%%  Zur Darstellung des Euro-Symbols   %%
\usepackage{eurosym, wasysym}

%%   Fuer Bibtex nach APA Style (American Psychology Association)   %%
%nun vawi-main.tex

% Seiten im Querformat
\usepackage{lscape}

% Code-Hervorhebung
% Quellcode
\usepackage{verbatim}            % Quellcode einbinden (\verbatiminput) standardpaket
\usepackage{moreverb} 
% PseudoCode
\usepackage[chapter]{algorithm}
\usepackage{algpseudocode}
%\usepackage{algorithmicx}

%%   intoc zur Aufnhame des Abkuerzungs- und Symbolverzeichnisses ins Inhaltsverzeichnis  
\usepackage[intoc]{nomencl}
\setlength{\nomlabelwidth}{.20\hsize}
%\renewcommand{\nomlabel}[1]{#1 \dotfill}
\setlength{\nomitemsep}{-\parsep}
\makenomenclature
\newcommand{\nomaltpreamble}{}
\newcommand{\nomaltpostamble}{}
\newcommand{\usetwonomenclatures}{\nomenclature[\switchnomitem]{}{}}
\newcommand{\switchnomitem}{R}
\renewcommand{\nomgroup}[1]{%
\ifthenelse{\equal{#1}{\switchnomitem}}{\switchnomalt}{}}
\newcommand{\switchnomalt}{%
\end{thenomenclature}
\newpage
\renewcommand{\nomname}{\nomaltname}
\renewcommand{\nompreamble}{\nomaltpreamble}
\renewcommand{\nompostamble}{\nomaltpostamble}
\begin{thenomenclature}
}



%%   Hervorhebung der Abkuerzungsbuchstaben   %%
\usepackage[normalem]{ulem}
\newcommand{\m}[1]{\uline{#1}}

%%%%%%%%%%%%%%%%%%%%%%%%%%%%%%%%%%%%%%%%%%%%%%%%%%%%%%%%%%%%%%%%%%%%%%%%%%%%%%%%%%%%%%%%%%%
%%                                   COMMAND SETUP                                       %%
%%%%%%%%%%%%%%%%%%%%%%%%%%%%%%%%%%%%%%%%%%%%%%%%%%%%%%%%%%%%%%%%%%%%%%%%%%%%%%%%%%%%%%%%%%%
\newcommand{\HRule}{\rule{\linewidth}{0.5mm}}

%#1 Breite
%#2 Datei (liegt im image Verzeichnis)
%#3 Beschriftung
%#4 Label fuer Referenzierung
\newcommand{\image}[4]{
\begin{figure}[H]
\centering
\includegraphics[width=#1]{image/#2}
\caption{#3}
\label{#4}
\end{figure}
}

%#1 Datei (liegt im graphic Verzeichnis)
%#2 Beschriftung
%#3 Label fuer Referenzierung
\newcommand{\tikzimage}[3]{%
\begin{figure}[H]%
\centering%
\input{graphic/#1.tikz}%
\caption{#2}%
\label{#3}%
\end{figure}
}

%#1 Datei (liegt im graphic Verzeichnis)
%#2 Beschriftung
%#3 Label fuer Referenzierung
%#4 Skalierungsfaktor
\newcommand{\scaletikzimage}[4]{%
\begin{figure}[H]%
\centering%
\scalebox{#4}{%
\input{graphic/#1.tikz}}%
\caption{#2}%
\label{#3}%
\end{figure}
}

%#1 Breite
%#2 Höhe
%#2 Datei (liegt im image Verzeichnis)
%#3 Beschriftung
%#4 Label fuer Referenzierung
\newcommand{\imagebh}[5]{
\begin{figure}[H]
\centering
\includegraphics[width=#1, height=#2]{image/#3}
\caption{#4}
\label{#5}
\end{figure}
}

%#1 Breite
%#2 Datei (liegt im image Verzeichnis)
%#3 zugehörige Bildunterschrift
%#4 Beschriftung
%#5 Label fuer Referenzierung
\newcommand{\mathimage}[5]{
\begin{figure}[H]
\centering
\includegraphics[width=#1]{image/#2}\\
#3
\caption{#4}
\label{#5}
\end{figure}
}

%#1 algorithm name
%#2 algorithm label
%#3 file name in code-folder
\newcommand{\pseudo}[3]{%
\small%
\begin{algorithm}[H]%
\caption{#1}%
\label{#2}%
\input{code/#3.tex}%
\end{algorithm}%
\normalsize%
}

%#1 Text der als todo dargestellt werden soll
\newcommand{\todo}[1]{
\begin{quote}
\textcolor{red}{\textbf{TODO: #1}}
\end{quote}
}

\newcommand{\rimage}[2]{
\begin{figure}[H]
\centering
#1
%\caption{#2}
\end{figure}
}

\newcommand \rack {
       {\LARGE $\square$}
}

\newcommand \deftab
{\hspace{1.5cm}\=abcdfffefghijk\hspace{1cm}\=1\hspace{1.5cm}\=1\hspace{1.5cm}\=1\hspace{1.5cm}\=1\hspace{1.5cm}\=1\hspace{1.5cm}\=asdjadj\kill}

\newcommand \einsbisfuenf
{\> {\bf -2} \> {\bf -1} \> {\bf 0} \> {\bf 1} \> {\bf 2} \>}

% #1 videofile
% #2 scalefactor
\newcommand{\video}[2]{%
\includemovie[text={\includegraphics[scale=#2]{praesi/video/#1.png}}, autoplay, mouse=true, repeat=1]{}{}{praesi/video/#1.swf}}


% Background Picture
\usepackage{eso-pic}
\newcommand\BackgroundPic{
\put(-15,0){
\parbox[b][\paperheight]{\paperwidth}{%
\vfill
\centering
\includegraphics[width=\paperwidth, height=\paperheight,
keepaspectratio]{image/deckblatt.pdf}%
\vfill
}}}

\newcommand\HeaderPic{
%\begin{picture}(0,0)%
\put(290,771){\mbox{\scriptsize{\sffamily \color{vawiblue}\textbf{Virtueller Weiterbildungsstudiengang Wirtschaftsinformatik\color{black}}}}}%

\put(0,0){
\parbox[b][\paperheight]{\paperwidth}{%
\centering
\vspace{22.4mm}
\noindent\makebox[\linewidth]{\color{vawiblue}\rule{2\paperwidth}{2.5pt}\color{black}}
\vfill
}}
}

%%%%%%%%%%%%%%%%%%%%%%%%%%%%%%%%%%%%%%%%%%%%%%%%%%%%%%%%%%%%%%%%%%%%%%%%%%%%%%%%%%%%%%%%%%%
%%                                 Kopf und Fusszeilen %%
%%%%%%%%%%%%%%%%%%%%%%%%%%%%%%%%%%%%%%%%%%%%%%%%%%%%%%%%%%%%%%%%%%%%%%%%%%%%%%%%%%%%%%%%%%%
\usepackage[automark, singlespacing=true]{scrlayer-scrpage}            % Kopf und Fusszeilen-Layout
\usepackage{scrhack}
\renewcommand*{\chapterpagestyle}{scrheadings} % Auch Chapters mit Kopfzeile
\renewcommand{\headfont}{\normalfont\sffamily\itshape}    % Kolumnentitel serifenlos
\renewcommand{\pnumfont}{\normalfont\sffamily}    % Seitennummern serifenlos
\pagestyle{scrheadings}
\addtolength{\headheight}{.8cm} %Abstand zu text
\addtolength{\textheight}{-0.8cm}

\ihead[]{
\AddToShipoutPicture*{\HeaderPic}
}
\ohead[]{\includegraphics[width=24mm]{image/vawiheaderlogo.png}}
%\chead[]{~\\[2mm]
%\hfill \scriptsize \color{vawiblue}\textbf{Virtueller Weiterbildungsstudiengang Wirtschaftsinformatik\color{black}}}
\chead[]{}
\ofoot[]{}                       % Seitennummern in der Fusszeile loeschen
\cfoot[]{}                       % Seitennummern in der Fusszeile loeschen
\ifoot{~
\makebox[145mm][r]{%
\begin{minipage}[r]{200mm}%
\color{vawiblue}\rule{192mm}{2pt}\color{black}\\[\dimexpr-\baselineskip+1mm+1.5pt]\color{vawigrayblue}\rule{192mm}{2pt}\color{black}\, \pagemark%
\end{minipage}%
}
\hfill
\begin{minipage}{1mm}
\color{vawiblue}\rule{40mm}{2pt}\color{black}\\
[\dimexpr-\baselineskip+1mm+1.5pt]
\color{vawigrayblue}\rule{40mm}{2pt}\color{black}
\end{minipage}
} 